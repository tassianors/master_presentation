%-------------------------------------------------------------------------------------------------
\section{Introdu��o} 
%-------------------------------------------------------------------------------------------------

\frame{\frametitle{Introdu��o}
Identifica��o de sistemas pode ser dividido em 3 partes principais:

	\begin{itemize}
		\item O sistema real a ser identificado $\mathcal{S}$
			\begin{equation}
				\mathcal{S}:\;\; y(t)=G_0(q)u(t)+H_0(q)e(t)
			\label{eq:si_intro_true_system}
			\end{equation}

		\item A classe de modelos a ser utilizada na identifica��o $\mathcal{M}$
			\begin{equation}
				\mathcal{M}: \;\;\left \{ G(q, \theta), H(q, \theta) | \theta \in D_{\mathcal{M}} \right \}
			\label{eq:si_intro_model}
			\end{equation}

		\item Algum crit�rio para elencar qual modelo dentro da classe de modelos melhor
		consegue representar o sistema $\mathcal{S}$ nas propriedades escolhidas	
	\end{itemize}
}

\frame{\frametitle{Modelagem de sistemas n�o lineares}
  Existem diversas fam�lias de modelos para representar sistemas n�o lineares. Algumas delas s�o:

	\begin{itemize}
		\item Wiener e Hammerstein
		\item S�rie de Volterra
		\item Redes Neurais: multi camada e recorrentes
		\item Fun��es Radiais de Base
		\item Modelos NARMAX
	\end{itemize}

	\begin{block}{NARMAX}
	Nonlinear Autoregressive Moving Average Model with Exogenous Variables
	\begin{eqnarray}\nonumber
		y(t)&=&F [ y(t-1), ..., y(t-n_y), u(t-1), ... , u(t-n_u),\\
		&&  e(t),e(t-1), ... , e(t-n_e) ]
	\nonumber
	\end{eqnarray}
	\end{block}
}

