%-------------------------------------------------------------------------------------------------
\section{Classes de Modelos}
%-------------------------------------------------------------------------------------------------
\subsection{Classes de modelos lineares} 
%-------------------------------------------------------------------------------------------------
\begin{frame}
\frametitle{Classes de modelos}
\framesubtitle{Sistemas lineares - SISO de tempo discreto}

\begin{columns}
\column{0.6\textwidth}
	\begin{block}{Classe de modelos gen�rica}
	\begin{equation}
		A(z)y(t)=\frac{B(z)}{F(z)}u(t)+\frac{C(z)}{D(z)}e(t)
		\nonumber
	\end{equation}
	\end{block}
	onde:
	\begin{align*}
		A(z, \theta) &= 1+a_1z^{-1}+...+a_{na}z^{-n_a} \\ 
		B(z, \theta) &= b_1z^{-1}+...+b_{nb}z^{-n_b}\\ 
		C(z, \theta) &= 1+c_1z^{-1}+...+c_{nc}z^{-n_c}\\ 
		D(z, \theta) &= 1+d_1z^{-1}+...+d_{nd}z^{-n_d}\\ 
		F(z, \theta) &= 1+f_1z^{-1}+...+f_{nf}z^{-n_f}
	\end{align*}
	
\column{0.4\textwidth}
\begin{small}
\begin{table}
\begin{center}
\label{table:si_modeling_models}
\begin{tabular}{cl}
\hline
        Polin�mios $\neq 1$ & Nome   \\
\hline
        B                 & FIR  	      \\ 
        AB                & ARX           \\ 
        ABC               & ARMAX         \\ 
        AC                & ARMA          \\ 
        ABD               & ARARX         \\ 
        ABCD              & ARARMAX       \\ 
        BF                & OE            \\ 
        BFCD              & Box-Jenkins   \\
\hline
\end{tabular}
\end{center}
\end{table}
\end{small}
\end{columns}
\end{frame}

%-------------------------------------------------------------------------------------------------
\subsection{Classes de modelos n�o lineares} 
%-------------------------------------------------------------------------------------------------
\begin{frame}
\frametitle{Classes de modelos}
\framesubtitle{Sistemas n�o lineares}

\begin{itemize}
  \item Muitos s�o os tipos de classes de modelos para sistemas n�o lineares.
  \item Eles podem ser divididos em dois t�pos principais:
\begin{itemize}
  \item N�o linearidades est�ticas
  \item N�o linearidades din�micas  
\end{itemize}
	\item Diversas classes de modelos se diferenciam pela escolha da base que ir� representar o sistema.
\end{itemize}

\end{frame}

%-------------------------------------------------------------------------------------------------
\begin{frame}
\frametitle{Classes de modelos - Wiener e Hammerstein}
\framesubtitle{Sistemas n�o lineares}

\begin{columns}
\column{0.5\textwidth}
\begin{block}{Representa n�o linearidades est�ticas}
\textcolor{blue}{Wiener}: N�o linearidade acoplada na sa�da do sistema

\textcolor{blue}{Hammerstein}: N�o linearidade acoplada na entrada do sistema
\end{block}

\column{0.5\textwidth}
\begin{figure}[htbp]
\center
\scalebox{0.6} % Change this value to rescale the drawing.
{
\begin{pspicture}(0,-1.6)(9.439062,1.6)
\usefont{T1}{ptm}{m}{n}
\rput(0.52453125,1.31){$u(t)$}
\usefont{T1}{ptm}{m}{n}
\rput(3.9557812,1.31){$f(u(t))$}
\usefont{T1}{ptm}{m}{n}
\rput(7.554531,1.35){$y(t)$}
\usefont{T1}{ptm}{m}{n}
\rput(5.8759375,1.15){Modelo}
\usefont{T1}{ptm}{m}{n}
\rput(5.7834377,0.75){Linear}
\usefont{T1}{ptm}{m}{n}
\rput(2.1957812,1.03){$f(\cdot)$}
\psframe[linewidth=0.04,dimen=outer](3.0,1.6)(1.2,0.4)
\psframe[linewidth=0.04,dimen=outer](6.8,1.6)(5.0,0.4)
\psline[linewidth=0.04cm,arrowsize=0.05291667cm 2.0,arrowlength=1.4,arrowinset=0.4]{->}(3.0,1.0)(5.0,1.0)
\psline[linewidth=0.04cm,arrowsize=0.05291667cm 2.0,arrowlength=1.4,arrowinset=0.4]{->}(0.0,1.0)(1.2,1.0)
\psline[linewidth=0.04cm,arrowsize=0.05291667cm 2.0,arrowlength=1.4,arrowinset=0.4]{->}(6.8,1.0)(8.0,1.0)
\usefont{T1}{ptm}{m}{n}
\rput(0.52453125,-0.69){$u(t)$}
\usefont{T1}{ptm}{m}{n}
\rput(3.9357812,-0.69){$z(t)$}
\usefont{T1}{ptm}{m}{n}
\rput(2.0759375,-0.85){Modelo}
\usefont{T1}{ptm}{m}{n}
\rput(1.9834375,-1.25){Linear}
\usefont{T1}{ptm}{m}{n}
\rput(5.9957814,-0.99){$f(\cdot)$}
\psframe[linewidth=0.04,dimen=outer](3.0,-0.4)(1.2,-1.6)
\psframe[linewidth=0.04,dimen=outer](6.8,-0.4)(5.0,-1.6)
\psline[linewidth=0.04cm,arrowsize=0.05291667cm 2.0,arrowlength=1.4,arrowinset=0.4]{->}(3.0,-1.0)(5.0,-1.0)
\psline[linewidth=0.04cm,arrowsize=0.05291667cm 2.0,arrowlength=1.4,arrowinset=0.4]{->}(0.0,-1.0)(1.2,-1.0)
\psline[linewidth=0.04cm,arrowsize=0.05291667cm 2.0,arrowlength=1.4,arrowinset=0.4]{->}(6.8,-1.0)(8.0,-1.0)
\usefont{T1}{ptm}{m}{n}
\rput(8.204532,-0.65){$y(t)=f(z(t))$}
\end{pspicture} 
}
\caption{Acima: modelo de Hammerstein. Abaixo: Modelo de Wiener.}
\label{fig:nl_models_hammerstein_wiener}
\end{figure}
\end{columns}

\end{frame}
%-------------------------------------------------------------------------------------------------
\begin{frame}
\frametitle{Classes de modelos}
\framesubtitle{Sistemas n�o lineares}

\begin{itemize}
  \item Voltera: Relaciona valores passados da entrada com o valor atual da sa�da.
  	\begin{equation}
		y(t)=\sum_{j=1}^{\infty}\int_{-\infty}^{\infty}\cdots \int_{-\infty}^{\infty}
		h_j(\tau_1, ... ,\tau_j) \prod_{i=1}^{j}u(t-\tau_i)d\tau_i
	\nonumber
	\end{equation}
  \item Redes Neurais: 
  \begin{equation}
	x=f\left ( \sum_{j=1}^{n}\omega_j x_j +b \right )
	\label{eq:nl_models_neural}
  \end{equation}
  \begin{itemize}
    \item Redes neurais multi camadas
    \item Redes neurais recorrente
  \end{itemize}
  \item Fun��es radiais de base: Casos particulares de redes neurais, mas lineares nos parametros $\omega_i$.
  \begin{equation}
	f(y)=\omega_0+\sum_{i}\omega_i \phi (\left \| y-c_i \right \|)
	\label{eq:nl_models_rbf}
  \end{equation}	
\end{itemize}

\end{frame}
%-------------------------------------------------------------------------------------------------
\begin{frame}
\frametitle{Classes de modelos - NARMAX}
\framesubtitle{Sistemas n�o lineares}

\begin{block} {Nonlinear Autoregressive Moving Average model with eXogenous variables}
\begin{equation}
y(t)=\theta^T\Phi_{nl}(y, u, e)
\label{eq:nlin_models_narmax_generic}
\end{equation}
onde $\Phi_{nl}(\cdot)$ denota um campo vetorial que depende dos valores passados de $y(t)$ e presente e passados de
$u(t)$ e $e(t)$; $\theta$ � o vetor de par�metros a ser identificado.
\end{block}

\end{frame}