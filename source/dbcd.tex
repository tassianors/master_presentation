%-------------------------------------------------------------------------------------------------
\section{Projeto de controladores baseados em dados}
%-------------------------------------------------------------------------------------------------

\begin{frame}
\frametitle{Projeto de controladores baseados em dados}
\framesubtitle{Defini��es}

\begin{block}{Sistema b�sico de controle}
\begin{figure}
\center
% Generated with LaTeXDraw 2.0.8
% Wed Jun 20 23:11:16 BRT 2012
% \usepackage[usenames,dvipsnames]{pstricks}
% \usepackage{epsfig}
% \usepackage{pst-grad} % For gradients
% \usepackage{pst-plot} % For axes
\scalebox{0.7} % Change this value to rescale the drawing.
{
\begin{pspicture}(0,-1.8192188)(9.851875,1.8192188)
\pscircle[linewidth=0.04,dimen=outer](1.431875,-0.79921883){0.2}
\psframe[linewidth=0.04,dimen=outer](4.431875,-0.3992188)(2.631875,-1.1992189)
\psframe[linewidth=0.04,dimen=outer](7.231875,-0.3992188)(5.631875,-1.1992189)
\pscircle[linewidth=0.04,dimen=outer](8.431875,-0.79921883){0.2}
\psline[linewidth=0.04cm,arrowsize=0.05291667cm 2.0,arrowlength=1.4,arrowinset=0.4]{->}(0.031875,-0.79921883)(1.231875,-0.79921883)
\psline[linewidth=0.04cm,arrowsize=0.05291667cm 2.0,arrowlength=1.4,arrowinset=0.4]{->}(1.631875,-0.79921883)(2.631875,-0.79921883)
\psline[linewidth=0.04cm,arrowsize=0.05291667cm 2.0,arrowlength=1.4,arrowinset=0.4]{->}(4.431875,-0.79921883)(5.631875,-0.79921883)
\psline[linewidth=0.04cm,arrowsize=0.05291667cm 2.0,arrowlength=1.4,arrowinset=0.4]{->}(7.231875,-0.79921883)(8.231875,-0.79921883)
\psline[linewidth=0.04cm,arrowsize=0.05291667cm 2.0,arrowlength=1.4,arrowinset=0.4]{->}(8.43,0.39921868)(8.431875,-0.5992188)
\psline[linewidth=0.04cm,arrowsize=0.05291667cm 2.0,arrowlength=1.4,arrowinset=0.4]{->}(8.631875,-0.79921883)(9.831875,-0.79921883)
\psline[linewidth=0.04cm,arrowsize=0.05291667cm 2.0,arrowlength=1.4,arrowinset=0.4]{<-}(1.431875,-0.99921876)(1.431875,-1.7992188)
\psline[linewidth=0.04cm](1.431875,-1.7992188)(9.231875,-1.7992188)
\psline[linewidth=0.04cm](9.231875,-1.7992188)(9.231875,-0.79921883)
\usefont{T1}{ptm}{m}{n}
\rput(1.0698436,-0.48921886){+}
\usefont{T1}{ptm}{m}{n}
\rput(8.069845,-0.48921886){+}
\usefont{T1}{ptm}{m}{n}
\rput(8.069845,-1.0892189){+}
\usefont{T1}{ptm}{m}{n}
\rput(1.6339064,-1.0892189){-}
\usefont{T1}{ptm}{m}{n}
\rput(0.441875,-0.49921885){\small $r(t)$}
\usefont{T1}{ptm}{m}{n}
\rput(3.561875,-0.79921883){\small $C(z, \theta)$}
\usefont{T1}{ptm}{m}{n}
\rput(6.331875,-0.8192189){\small $G_0(z)$}
\usefont{T1}{ptm}{m}{n}
\rput(5.081875,-0.49921885){\small $u(t)$}
\usefont{T1}{ptm}{m}{n}
\rput(1.941875,-0.49921885){\small $\epsilon (t)$}
\usefont{T1}{ptm}{m}{n}
\rput(8.851875,0.12078118){\small $\nu(t)$}
\usefont{T1}{ptm}{m}{n}
\rput(9.271875,-0.49921885){\small $y(t)$}
\psframe[linewidth=0.04,dimen=outer](9.231875,1.2007811)(7.631875,0.40078118)
\usefont{T1}{ptm}{m}{n}
\rput(8.411875,0.7807812){\small $H_0(z)$}
\psline[linewidth=0.04cm](8.43,1.1992186)(8.43,1.7992188)
\usefont{T1}{ptm}{m}{n}
\rput(8.871875,1.5807811){\small $e(t)$}
\end{pspicture} 
}
\caption{Representa��o de um sistema de controle em malha fechada, com ru�do aditivo na sa�da.}
\label{fig:vrft_db_control_loop}
\end{figure}
\end{block}

\begin{equation}
T(z, \theta)=\frac{C(z,\theta)G_0(z)}{1+C(z,\theta)G_0(z)}
\label{eq:vrft_db_closed_loop}
\end{equation}

\end{frame}

%-------------------------------------------------------------------------------------------------
%-------------------------------------------------------------------------------------------------

\begin{frame}
\frametitle{Projeto de controladores baseados em dados}
\framesubtitle{Crit�rios de perfomance}

\begin{itemize}
  \item Seguimento de refer�ncia,
  \item Rejei��o ao ru�do e 
  \item Uso reduzido de esfor�o de controle. 
\end{itemize}

Para o objetivo de \textcolor{blue}{seguimento de refer�ncia}, a performance pode ser avaliada pela norma:
\begin{equation}
J_y(\theta)\overset{\underset{\mathrm{\Delta}}{\,}}{=}  \bar{E} \left [ y(t)-y_d(t) \right ]^2 = \bar{E}\left [
(T(z,\theta)-T_d(z))r(t) \right ]^2
\nonumber
\end{equation}

\begin{block}{Controlador ideal - Seguimento de refer�ncia}
\begin{equation}
C_d(z)=\frac{T_d(z)}{G_0(z)(1-T_d(z))}
\nonumber
\end{equation}
\end{block}
\end{frame}

%-------------------------------------------------------------------------------------------------
%-------------------------------------------------------------------------------------------------
%-------------------------------------------------------------------------------------------------
\subsection{Refer�ncia Virtual}
%-------------------------------------------------------------------------------------------------

\begin{frame}
\frametitle{Refer�ncia Virtual para identifica��o de controladores}
\framesubtitle{M�todo VRFT}

\begin{itemize}
  \item M�todo VRFT utiliza refer�ncia virtual para obten��o dos sinais necess�rios � identifica��o. 
\end{itemize}

\begin{figure}
\center
\scalebox{0.7} % Change this value to rescale the drawing.
{
\begin{pspicture}(0,-1.4292188)(9.02,1.4692187)
\pscircle[linewidth=0.04,linestyle=dashed,dash=0.16cm 0.16cm,dimen=outer](1.4,0.97078127){0.2}
\psframe[linewidth=0.04,linestyle=dashed,dash=0.16cm 0.16cm,dimen=outer](4.8,1.3707813)(3.0,0.57078123)
\psframe[linewidth=0.04,dimen=outer](7.6,1.3707813)(6.0,0.57078123)
\psline[linewidth=0.04cm,arrowsize=0.05291667cm 2.0,arrowlength=1.4,arrowinset=0.4]{->}(0.0,0.97078127)(1.2,0.97078127)
\psline[linewidth=0.04cm,linestyle=dashed,dash=0.16cm 0.16cm,arrowsize=0.05291667cm 2.0,arrowlength=1.4,arrowinset=0.4]{->}(1.6,0.97078127)(3.0,0.97078127)
\psline[linewidth=0.04cm,arrowsize=0.05291667cm 2.0,arrowlength=1.4,arrowinset=0.4]{->}(4.8,0.97078127)(6.0,0.97078127)
\psline[linewidth=0.04cm](7.6,0.97078127)(9.0,0.97078127)
\psline[linewidth=0.04cm,linestyle=dashed,dash=0.16cm 0.16cm,arrowsize=0.05291667cm 2.0,arrowlength=1.4,arrowinset=0.4]{<-}(1.4,0.7707813)(1.4,-0.02921875)
\psline[linewidth=0.04cm,linestyle=dashed,dash=0.16cm 0.16cm](1.4,-0.02921875)(8.4,-0.02921875)
\psline[linewidth=0.04cm,linestyle=dashed,dash=0.16cm 0.16cm](8.4,-0.02921875)(8.4,0.97078127)
\usefont{T1}{ptm}{m}{n}
\rput(1.1126562,1.2807813){+}
\usefont{T1}{ptm}{m}{n}
\rput(1.6473438,0.68078125){-}
\usefont{T1}{ptm}{m}{n}
\rput(0.47,1.2707813){\small $\bar{r}(t)$}
\usefont{T1}{ptm}{m}{n}
\rput(3.99,0.97078127){\small $C(z, \theta)$}
\usefont{T1}{ptm}{m}{n}
\rput(6.76,0.9507812){\small $G_0(z)$}
\usefont{T1}{ptm}{m}{n}
\rput(5.51,1.2707813){\small $u(t)$}
\usefont{T1}{ptm}{m}{n}
\rput(2.25,1.2707813){\small $\epsilon (t)$}
\usefont{T1}{ptm}{m}{n}
\rput(8.3,1.2707813){\small $y(t)$}
\psframe[linewidth=0.04,dimen=outer](6.0,-0.62921876)(3.8,-1.4292188)
\usefont{T1}{ptm}{m}{n}
\rput(4.91,-1.0492188){\small $T_d^{-1}(z)$}
\psline[linewidth=0.04cm](9.0,0.97078127)(9.0,-1.0292188)
\psline[linewidth=0.04cm](9.0,-1.0292188)(6.0,-1.0292188)
\psline[linewidth=0.04cm](3.8,-1.0292188)(0.0,-1.0292188)
\psline[linewidth=0.04cm](0.0,-1.0292188)(0.0,0.97078127)
\end{pspicture} 
}
\label{fig:vrft_method_cl}
\end{figure}

\begin{itemize}
  \item O m�todo VRFT faz com que a fun��o custo a ser minimizada seja quadr�tica em $\theta$, n�o recaindo em m�nimos
  locais.
\end{itemize}

\end{frame}
%-------------------------------------------------------------------------------------------------
%-------------------------------------------------------------------------------------------------

\begin{frame}
\frametitle{Refer�ncia Virtual para identifica��o de controladores}
\framesubtitle{M�todo VRFT - fun��o custo de um sistem hipot�tico}

\begin{figure}
\center
% Generated with LaTeXDraw 2.0.8
% Mon Jul 02 22:05:15 BRT 2012
% \usepackage[usenames,dvipsnames]{pstricks}
% \usepackage{epsfig}
% \usepackage{pst-grad} % For gradients
% \usepackage{pst-plot} % For axes
\scalebox{0.55} % Change this value to rescale the drawing.
{
\begin{pspicture}(0,-4.62)(11.799063,4.62)
\psbezier[linewidth=0.04](0.9,3.3)(2.1,3.2)(2.0225368,2.3742967)(2.1565964,1.5166456)(2.290656,0.6589943)(2.616248,-1.3127834)(3.008854,-0.5402044)(3.4014597,0.23237456)(3.6584003,0.8693678)(3.811938,0.20167358)(3.9654756,-0.4660206)(4.5199084,-3.4794219)(5.1017394,-3.533306)(5.68357,-3.5871902)(8.0,1.8)(9.5,3.2)
\psbezier[linewidth=0.04,linestyle=dashed,dash=0.16cm 0.16cm](0.4,2.2)(2.2,-2.6)(3.960841,-3.4989967)(5.12,-3.54)(6.279159,-3.5810034)(9.3,-1.9)(10.6,1.9)
\psline[linewidth=0.04cm,arrowsize=0.05291667cm 4.0,arrowlength=1.4,arrowinset=0.4]{<-}(1.1,4.6)(1.2,-4.6)
\psline[linewidth=0.04cm,arrowsize=0.05291667cm 4.0,arrowlength=1.4,arrowinset=0.4]{->}(0.0,-3.9)(11.3,-3.9)
\psline[linewidth=0.04cm,linestyle=dotted,dotsep=0.16cm](5.1,-4.0)(5.0,3.1)
\usefont{T1}{ppl}{m}{n}
\rput(5.114531,-4.19){$\theta^*$}
\usefont{T1}{ppl}{m}{n}
\rput(11.014531,-4.19){$\theta$}
\usefont{T1}{ppl}{m}{n}
\rput(0.5545313,4.01){$J(\theta)$}
\usefont{T1}{ppl}{m}{n}
\rput(2.5545313,2.71){$J_y$}
\usefont{T1}{ppl}{m}{n}
\rput(3.2145312,-1.79){$J_{VR}$}
\end{pspicture} 
}
\caption{O valor $\theta^*$ � o ponto de m�nimo de ambas as fun��es custo, logo, minimizando a fun��o custo
$J_{VR}(\theta)$ � o equivalente a minimizar $J_y(\theta)$ sob condi��es ideais.}
\label{fig:vrft_cost_functions}
\end{figure}

\end{frame}